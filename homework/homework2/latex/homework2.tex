\documentclass{note}

\usepackage{tikz}
\usepackage{pgfplots}

\usetikzlibrary{external}
% \tikzexternalize

\usepgfplotslibrary{fillbetween}

\newtheorem{theorem}{定理}
\theoremstyle{definition}
\newtheorem{definition}{定义}

\NewDocumentCommand\optmodule{sO{}mmm}{%
    \IfBooleanTF{#1}{%
        \begin{equation*}
            \ifthenelse{\equal{#2}{}}{}{(#2)\quad}
            \begin{aligned}
                \begin{cases}
                    #3 \quad & #4 \\
                    \mathrm{s.t.} \quad #5
                \end{cases}
            \end{aligned}
        \end{equation*}
    }{%
        \begin{equation}
            \ifthenelse{\equal{#2}{}}{}{(#2)\quad}
            \begin{aligned}
                \begin{cases}
                    #3 \quad & #4 \\
                    \mathrm{s.t.} \quad #5
                \end{cases}
            \end{aligned}
        \end{equation}
    }%
}
\begin{document}
    \begin{center}
        {\Large\bfseries 组合优化与凸优化作业2}
        \vspace{1cm}

        胡冠宇
    \end{center}

    所有代码见附件或GitHub备份\footnote{\href{https://github.com/ikaroinory/convex-opt/tree/main/homework/homework2/src}{https://github.com/ikaroinory/convex-opt/tree/main/homework/homework2/src}}.

    \foreach \i in {1,...,15} {
        \input{answers/\i.tex}
    }

    % \phantomsection
    % \addcontentsline{toc}{section}{\refname}
    \bibliography{references}
\end{document}
