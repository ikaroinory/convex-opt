\section*{15}

共轭函数定义如下:
\begin{definition}
    \label{definition:15a}
    对于函数$f:\mathbb{R}^n\to\mathbb{R}\cup\{-\infty,+\infty\}$, 其共轭函数$f^*$定义为
    \begin{equation*}
        f^*(\bm{y})=\sup_{x\in\mathbb{R}^n}\{\left<\bm{y},\bm{x}\right>-f(\bm{x})\},
    \end{equation*}
    其中$\left<\cdot,\cdot\right>$为Euclid内积.
\end{definition}

求解共轭函数一般有如下步骤:
\begin{enumerate}
    \item \textbf{构造优化问题}.
        由\cref{definition:15a}可知, 求解共轭函数是一个关于$\bm{x}$的优化问题.

    \item \textbf{求导}.
        若$f(x)$可微, 找到使得梯度为零的$x$, 即解方程
        \begin{equation*}
            \nabla_{\bm{x}}(\left<\bm{y},\bm{x}\right>-f(\bm{x}))=\bm{0}.
        \end{equation*}
    
    \item \textbf{凸性与Legendre变换}.
        若$f(\bm{x})$是严格凸函数, 求共轭函数的过程等价于Legendre变换, 即
        \begin{equation*}
            f^*(\bm{y})=\left<\bm{y},\bm{x}^*(\bm{y})\right>-f(\bm{x}^*(\bm{y})),
        \end{equation*}
        其中$\bm{x}^*(\bm{y})$是由$\bm{y}=\nabla f(\bm{x})$反解得到的.
\end{enumerate}

共轭函数在优化问题中扮演了关键角色, 尤其是在Lagrange对偶和Fenchel对偶中.
\begin{enumerate}
    \item \textbf{Lagrange对偶}.

        设优化问题为
        \optmodule*{\min_{\bm{x}}}{f(\bm{x})}{
            &g_i(\bm{x})\leq0, \\
            &h_j(\bm{x})=0,
        }
        则其对偶问题可通过Lagrange函数
        \begin{equation*}
            \mathcal{L}(\bm{x},\bm{\lambda},\bm{\mu})=f(\bm{x})+\sum_{i}\lambda_ig_i(\bm{x})+\sum_{j}\mu_jh_j(\bm{x})
        \end{equation*}
        用共轭函数表示目标函数, 从而推导对偶问题.
    
    \item \textbf{Fenchel对偶}.

        给定函数$f(\bm{x})$和$g(\bm{y})$, 考虑优化问题
        \begin{equation*}
            \min_{\bm{x}} f(\bm{x})+g(\bm{Ax}),
        \end{equation*}
        其Fenchel对偶问题为
        \begin{equation*}
            \max_{\bm{y}} f^*(-\bm{A}^\mathrm{T}\bm{y})-g^*(\bm{y}),
        \end{equation*}
        其中$f^*$和$g^*$分别为$f$和$g$的共轭函数.
\end{enumerate}

共轭函数在优化理论中起到了桥梁作用, 将原始问题映射到其对偶问题, 从而提供了一种有效的分析工具。利用共轭函数, 可以深入研究优化问题的对偶结构, 如强对偶性条件, 对偶间隙以及对偶问题的解构特性.
在凸优化领域, 共轭函数广泛应用于凸对偶理论, Fenchel对偶以及拉格朗日对偶方法, 为求解复杂优化问题提供了重要的理论支撑.
