\section*{9}

无约束优化问题形式为
\begin{equation*}
    \min_{\bm{x}\in\mathbb{R}^n} ~ f(\bm{x}),
\end{equation*}
其中$f(\bm{x})$为目标函数, 其定义在整个空间内, 且无其他约束条件.

无约束优化问题的求解通常有两类方法
\begin{enumerate}
    \item \textbf{梯度类方法}.

        梯度类方法利用梯度或Hessian矩阵信息, 来选取合适的搜索方向, 即
        \begin{equation*}
            \bm{x}^{(k+1)}=\bm{x}^{(k)}+\lambda_k\nabla f\left(\bm{x}^{(k)}\right).
        \end{equation*}
        常见的梯度类方法有最速下降法, 牛顿法, 共轭梯度法等.

    \item \textbf{搜索方法}.

        搜索方法不使用任何梯度或Hessian矩阵信息, 直接根据约束在空间中挑选线性无关的方向搜索.
        常见的搜索方法有模式搜索, Rosenbrock法, 单纯形搜索, Powell方法等.
\end{enumerate}

求解无约束优化问题通常有两个步骤: 
\begin{enumerate}
    \item 确定搜索方向.
    \item 确定搜索步长.
\end{enumerate}

非凸优化难以求解, 因此在理论和工程中常希望将其转化为凸问题. 常用方法有:
\begin{enumerate}
    \item 松弛.

        将原问题的非凸部分用凸函数近似或放松.

    \item 凸包方法.

    对非凸函数取其最紧凸上界, 构成凸优化问题.

    \item 重参数化或替代变量.

        通过变量替换, 使目标函数或约束变凸.

    \item 局部凸化.

        将非凸函数在某个邻域内近似为凸函数.

    \item 正则化与惩罚项.

    加入凸正则项, 引导目标函数变得更凸.
\end{enumerate}
