原问题与其对偶问题之间的核心联系体现在对偶定理.

\begin{theorem}[弱对偶定理]
    对于任何原问题的可行解$\bm{x}$及其对偶问题的可行解$\bm{y}$, 总有
    \begin{equation*}
        \bm{c}^\mathrm{T}\bm{x}\leq\bm{b}^\mathrm{T}\bm{y}
    \end{equation*}
\end{theorem}

\begin{theorem}[强对偶定理]
    若原问题及其对偶问题均有可行解且分别为$\bm{x}^*$和$\bm{y}^*$, 则
    \begin{equation*}
        \bm{c}^\mathrm{T}\bm{x}^*=\bm{b}^\mathrm{T}\bm{y}^*
    \end{equation*}
\end{theorem}

原问题与其对偶问题之间的区别在于:
\begin{itemize}
    \item
        目标函数的区别
        \begin{itemize}
            \item 原问题是最大化问题$\max\bm{c}^\mathrm{T}\bm{x}$.
            \item 其对偶问题是最大化问题$\min\bm{b}^\mathrm{T}\bm{y}$.
        \end{itemize}

    \item
        约束条件的转换
        \begin{itemize}
            \item 原问题的约束右端项$\bm{b}$变成其对偶目标函数的系数.
            \item 原问题的目标函数系数$\bm{c}$变成其对偶约束的右端项.
            \item 原问题的约束方向$\leq$变成对偶问题的$\geq$.
        \end{itemize}

    \item 变量的物理意义

        在资源分配问题中, 原问题变量$\bm{x}$表示实际分配方案, 而其对偶变量$\bm{y}$可以解释为资源的影子价格, 表示单位资源的价值.

    \item 计算难度

        若原问题有$n$个变量和$m$个约束, 当$m\ll n$时求解对偶问题可能更简单.
\end{itemize}
