\textbf{必要性}:
若$\bm{x}=\begin{pmatrix}x_1 & \cdots & x_m & 0 & \cdots 0\end{pmatrix}^\mathrm{T}\in\mathbb{R}^n$为基本可行解, 且$x_1,\cdots,x_m\geq0$, 对应的$\bm{B}=\begin{pmatrix}\bm{P}_1 & \bm{P}_2 & \cdots & \bm{P}_m\end{pmatrix}$.
因为$\bm{x}$是可行解, 设其中正分量的个数为$k$, 则$k\leq m$, 因此有
\begin{equation*}
    \begin{pmatrix}
        \bm{P}_1 & \bm{P}_2 & \cdots & \bm{P}_k
    \end{pmatrix}
    \subseteq
    \begin{pmatrix}
        \bm{P}_1 & \bm{P}_2 & \cdots & \bm{P}_m
    \end{pmatrix}.
\end{equation*}
又因为$\bm{P}_1, \bm{P}_2, \cdots, \bm{P}_m$线性无关, 因此$\bm{P}_1, \bm{P}_2, \cdots, \bm{P}_k$线性无关.


\textbf{充分性}:
若$\bm{x}=\begin{pmatrix}x_1 & \cdots & x_k & 0 & \cdots 0\end{pmatrix}^\mathrm{T}\in\mathbb{R}^n$为可行解, $x_1,\cdots,x_k>0$, 且$\bm{P}_1, \bm{P}_2, \cdots, \bm{P}_m$线性无关, $\mathrm{rank}(\bm{A})=m$, 则显然有$k\leq m$.

\begin{itemize}
    \item
        当$k=m$时, 显然$\bm{x}$是基本可行解.

    \item
        当$k<m$时, 有$\bm{P}_1, \bm{P}_2, \cdots, \bm{P}_k$线性无关. 因为$\mathrm{rank}(\bm{A})=m$, 所以必定可以从$\bm{P}_{k+1}, \bm{P}_{k+2}, \cdots, \bm{P}_n$中选出$m-k$个线性无关向量, 与$\bm{P}_1, \bm{P}_2, \cdots, \bm{P}_k$组成线性无关的向量组, 因此$\bm{x}$是基本可行解.
\end{itemize}
