\section*{4}

\begin{enumerate}[label = \alph*)]
    \item
        \begin{equation*}
            \nabla f(\bm{x})=
            \begin{pmatrix}
                4x_1-2x_2-4 \\
                -2x_1+4x_2-6
            \end{pmatrix},\quad
            \nabla^2 f(\bm{x})=
            \begin{pmatrix}
                4 & -2 \\
                -2 & 4
            \end{pmatrix}.
        \end{equation*}

        令$\nabla f(\bm{x})=\bm{0}$, 得到临界点$\bm{x}=\left(\frac{7}{3}, \frac{8}{3}\right)^\mathrm{T}$.
        经检验, 临界点$\bm{x}$不满足约束条件, 因此不是真正的极值点.

    \item
        \begin{align*}
            &\nabla g_1(\bm{x})=(1,1)^\mathrm{T}, \\
            &\nabla g_2(\bm{x})=(1,5)^\mathrm{T}, \\
            &\nabla g_3(\bm{x})=(-1,0)^\mathrm{T}, \\
            &\nabla g_4(\bm{x})=(0,-1)^\mathrm{T}.
        \end{align*}

    \item
        使用可行方向法求解题中优化问题, 过程见\cref{table:4}.
        因此最优解$\bm{x}^\star=\left(\frac{5}{2},\frac{11}{4}\right)^\mathrm{T}$, 最优值$f(\bm{x}^\star)=-\frac{101}{8}$.
        \begin{table}[ht]
            \centering
            \caption{可行方向法的求解过程}
            \label{table:4}
            \begin{tabular}{ccccc}
                \toprule
                $k$ & $\bm{x}_k$ & $\bm{g}$ & $\bm{d}$ & $\lambda$ \\
                \midrule
                0 & $(0,0)^\mathrm{T}$ & $(-2,-5,0,0)^\mathrm{T}$ & $\left(\frac{5}{4},\frac{3}{4}\right)$ & 2 \\
                1 & $\left(\frac{5}{2},\frac{3}{2}\right)^\mathrm{T}$ & $\left(2,5,-\frac{5}{2},-\frac{3}{2}\right)$ & $(0,1)$ & $\frac{5}{4}$ \\
                2 & $\left(\frac{5}{2},\frac{11}{4}\right)^\mathrm{T}$ & $\left(\frac{13}{4},\frac{45}{4},-\frac{5}{2},-\frac{11}{4}\right)$ & $(0,0)$ & 0 \\
                \bottomrule
            \end{tabular}
        \end{table}
\end{enumerate}
