\section*{9}

启发式优化算法是一类模拟自然或人工过程的优化方法, 通常不依赖于目标函数的梯度信息, 适用于求解非凸、非线性、离散或组合优化问题.
其基本思想是通过构造智能的搜索策略, 引导解空间的探索, 避免陷入局部最优, 并在有限计算资源下获得较优解.
典型方法包括遗传算法(GA)、粒子群优化(PSO)、蚁群算法(ACO)、模拟退火(SA)等.

点集匹配问题属于典型的组合优化问题, 其目标是找出两个点集之间的一一对应关系, 使得匹配后的点对在空间中尽量对齐.
设两个点集分别为$\{p_i\}_{i=1}^n$和$\{q_j\}_{j=1}^n$, 可通过建立一个匹配矩阵$\bm{M} \in \{0,1\}^{n\times n}$表示点对的对应关系, 使得整体匹配误差
\begin{equation*}
    \min_{\bm{M}\in\mathcal{P}}\sum_{i,j}m_{ij}\cdot\|p_i - q_j\|^2
\end{equation*}
最小, 其中 $\mathcal{P}$ 表示所有一一匹配的可行集合, $m_{ij}$为矩阵$\bm{M}$中第$i$行第$j$列的元素.
由于可行解空间的复杂度为$O(n!)$, 该问题为NP难问题.
可采用如下启发式方法求解:
\begin{itemize}
  \item \textbf{遗传算法(GA)}: 将匹配关系编码为排列, 定义匹配误差为适应度函数, 通过交叉、变异等操作在匹配空间中搜索最优解.
  \item \textbf{模拟退火(SA)}: 以初始匹配为起点, 逐步尝试小范围交换点对, 通过控制“温度”接受次优解, 避免陷入局部最优.
  \item \textbf{蚁群算法(ACO)}: 模拟蚂蚁在点对之间构建路径, 使用匹配误差引导信息素更新, 从而引导匹配决策.
\end{itemize}

拼图重构问题是另一类典型的组合优化问题, 其目标是将被打乱的图像碎片重新拼接为原图.
设有$n$个拼图块, 需确定其空间位置及方向, 使得相邻块边缘相似性最强.
该问题可转化为求解一个最优排列和方向组合的问题, 启发式算法的应用方式包括:

\begin{itemize}
  \item \textbf{遗传算法}: 编码方式可设计为块位置序列加方向集合, 适应度函数基于边缘相似度之和.
  \item \textbf{局部搜索 + 模拟退火}: 从随机初始拼图出发, 定义邻域操作(如两块交换、旋转某块), 以整体边缘匹配度为目标函数, 采用概率接受机制避免早期收敛.
  \item \textbf{粒子群优化(PSO)}: 将每个“粒子”视为一个拼图配置, 通过群体间位置和速度更新搜索高匹配度解.
\end{itemize}
