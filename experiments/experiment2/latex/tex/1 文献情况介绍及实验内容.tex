\section{文献情况介绍及实验内容}

工业控制系统(Industrial Control Systems,ICS)广泛应用于电力、制造、水利等关键基础设施,其安全性和稳定性至关重要。
一旦ICS受到攻击,可能会造成严重的损坏。
因此,ICS的异常检测是保障关键基础设施安全的核心任务。
传统异常检测方法主要关注单一域中的指标,如网络域中的网络流量或物理域中的传感器数据,但ICS中不同域(如传感器的物理状态、网络通信流量等)的行为存在强相关性,仅分析单一域难以全面识别异常。
随着ICS与互联网的深度融合,攻击者可通过多域协同攻击绕过传统单域检测机制。
例如,网络攻击可能导致传感器数据异常,但某些攻击仅影响物理设备而不改变网络流量。

现有的方法如基于RNN\cite{mandic2001recurrent}、GAN\cite{creswell2018generative}或基于单域图的神经网络模型无法有效建模跨域关联,导致检测精度不足和误报率高。
文章\cite{zhan2024anomaly}在GDN\cite{deng2021graph}的基础上提出了一种基于跨域表示学习的ICS异常检测方法(MGDN),该方法能够学习多域行为的联合特征,结合动态图结构学习与跨域交叉注意力机制,进一步提升多域异常检测的鲁棒性与实时性。

本文将进行如下工作:

第一,全面介绍MGDN模型的核心思想与整体架构。
具体包括多图构建方法、多图信息的融合策略、基于注意力机制的图卷积神经网络建模方式,以及最终融合输出的机制。
该部分将详细阐述如何通过构建多个表示不同语义关系的图,从而提升异常检测的表示能力和模型的泛化性能。

第二,深入探讨MGDN模型的优化策略。
本文将介绍多任务学习中的多梯度下降算法,给出其数学形式化定义,解释其在多任务场景下如何实现梯度的平衡与协调。
同时,本文将简要回顾多任务学习的发展历程。

第三,在SWaT数据集上开展模型性能评估实验。
通过与九种代表性的基线模型进行系统对比,从准确率、召回率、F1分数等多个维度评估MGDN的优势。此外,还将通过不同优化器的对比实验,进一步验证多梯度下降算法在MGDN中的关键作用和性能提升效果。
