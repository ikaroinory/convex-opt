\section{优化方法介绍}

\subsection{随机搜索法}

随机搜索法(Random Search Method)是一种无导数优化方法,其基本思想是在定义域内随机生成若干点,并选择函数值最小的作为当前最优解。
设优化问题为
\begin{equation*}
    \min_{\bm{x}\in\mathbb{R}^n} ~ f(\bm{x}) \text{。}
\end{equation*}
在第$k$次迭代中,随机生成点$\bm{x}_k'\in U(\bm{x}_k)$,其中$U(\bm{x}_k)$为$\bm{x}_k$的某一邻域。

则有迭代公式
\begin{equation*}
    \bm{x}_{k+1}=\bm{x}_k'
    \begin{cases}
        \bm{x}_k'\text{,} &f(\bm{x}_k')<f(\bm{x}_k) \text{,} \\
        \bm{x}_k\text{,} &f(\bm{x}_k')\geq f(\bm{x}_k) \text{。}
    \end{cases}
\end{equation*}
该方法通过不断试探更新寻找函数值更小的位置,而无需任何梯度信息。

\subsection{梯度下降法}

梯度下降法(Gradient Descent Method)利用目标函数的梯度信息,以梯度的负方向作为下降方向,即
\begin{equation*}
    \bm{x}_{k+1}=\bm{x}_k-\alpha_k\nabla f(\bm{x}_k) \text{,}
\end{equation*}
其中$\alpha_k>0$为学习率,$\nabla f(\bm{x}_k)$为目标函数在点$\bm{x}_k$处的梯度。

\subsection{次梯度下降法}

次梯度下降法(Subgradient Descent Method)用于优化非光滑凸函数。
若$f$在点$\bm{x}_k$处不可导,但存在次梯度$\bm{g}_k\in\partial f(\bm{x}_k)$, 则有
\begin{equation*}
    \bm{x}_{k+1}=\bm{x}_k-\alpha_k\bm{g}_k \text{,}
\end{equation*}
其中$\alpha_k>0$为学习率,$\partial f(\bm{x}_k)$为目标函数在点$\bm{x}_k$处的次梯度集合。

\subsection{共轭方向法}

共轭方向法(Conjugate Direction Method)用于二次函数
\begin{equation*}
    f(\bm{x})=\frac{1}{2}\bm{x}^\mathrm{T}\bm{Ax}-\bm{b}^\mathrm{T}\bm{x} \text{,}
\end{equation*}
其中矩阵$\bm{A}$对称正定。

共轭方向法需要构造一组关于矩阵$\bm{A}$的共轭方向$\{\bm{d}_i\}_{i=1}^n$,使得
\begin{equation*}
    \bm{d}_i\bm{A}\bm{d}_j=0 \text{,} i\ne j \text{,}
\end{equation*}
沿着这些方向进行线性搜索,即
\begin{equation*}
    \bm{x}_{k+1}=\bm{x}_k-\alpha_k\bm{d}_k \text{,}
\end{equation*}
其中$\alpha_k>0$为学习率,可通过一维搜索得到。

\subsection{共轭梯度法}

共轭梯度法(Conjugate Gradient Method)是一种特殊的共轭方向法,使用当前梯度和上一次方向构造共轭方向,避免计算$\bm{A}$。
对于二次函数,其迭代格式为:
\begin{align*}
    &\bm{x}_{k+1}=\bm{x}_k-\alpha_k\bm{d}_k \text{,} \\
    &\alpha_k=\frac{\bm{r}_k^\mathrm{T}\bm{r}_k}{\bm{d}_k^\mathrm{T}\bm{Ad}_k} \text{,} \\
    &\bm{d}_k=\bm{r}_k+\beta_{k-1}\bm{d}_{k-1} \text{,} \beta_{k-1}=\frac{\bm{r}_k^\mathrm{T}\bm{r}_k}{\bm{r}_{k-1}^\mathrm{T}\bm{r}_{k-1}} \text{,} \\
    &\bm{r}_k=\bm{b}-\bm{Ax}_k \text{,} \bm{r}_0=-\nabla f(\bm{x}_0) \text{。}
\end{align*}
