\section{结论}

工业控制系统与微服务架构的异常检测与根因分析是保障现代工业与互联网应用稳定性的核心问题。两篇论文通过跨领域图表示学习与潜在空间干预识别,为解决复杂系统中的多维度、异构性问题提供了创新方案。

ICS异常检测模型MGDN通过物理域与网络域数据的联合建模,克服了单领域分析的局限性。注意力机制与多任务优化的结合,使模型在保留领域特征的同时捕捉跨域关联。

文章\cite{tao2024diagnosing}在微服务系统性能诊断领域做出了重要贡献。它不仅提供了一种新的根因定位方法,还通过实际案例展示了该方法的有效性。尽管存在一些局限性,但文章也已经详尽写出,这位未来的研究工作进行了初步的规划和奠定了基础。MicroDig的出现,无疑为微服务系统的运维和性能优化提供了新的工具和思路,对于提高系统的可靠性和用户体验具有重要意义。此外,这项研究也为学术界和工业界提供了新的研究方向,特别是在微服务架构日益普及的背景下,如何有效地诊断和解决性能问题,成为了一个亟待解决的挑战。MicroDig的成功应用,为这一挑战提供了一个有力的解决方案,也为未来的研究和实践提供了宝贵的经验和启示。
